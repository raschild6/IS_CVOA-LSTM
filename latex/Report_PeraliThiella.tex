\def\year{2020}\relax
%File: formatting-instruction.tex
\documentclass[letterpaper]{article}% DO NOT CHANGE THIS
\usepackage{aaai20}  				% DO NOT CHANGE THIS
\usepackage{times}  				% DO NOT CHANGE THIS
\usepackage{helvet} 				% DO NOT CHANGE THIS
\usepackage{courier}  				% DO NOT CHANGE THIS
\usepackage[hyphens]{url}  			% DO NOT CHANGE THIS
\usepackage{graphicx} 				% DO NOT CHANGE THIS
\usepackage[backend=bibtex, style=numeric]{biblatex}
\urlstyle{rm} 						% DO NOT CHANGE THIS
\def\UrlFont{\rm}  					% DO NOT CHANGE THIS
\usepackage{graphicx}  				% DO NOT CHANGE THIS
\frenchspacing  					% DO NOT CHANGE THIS
\setlength{\pdfpagewidth}{8.5in}  	% DO NOT CHANGE THIS
\setlength{\pdfpageheight}{11in}  	% DO NOT CHANGE THIS
\usepackage{listings} %Per inserire codice ***
%\nocopyright
%PDF Info Is REQUIRED.
% For /Author, add all authors within the parentheses, separated by commas. No accents or commands.
% For /Title, add Title in Mixed Case. No accents or commands. Retain the parentheses.
 \pdfinfo{
/Title (Predizioni attraverso GA e LSTM basate sul Covid-19)
/Author (Luca Perali, Michele Thiella)
} %Leave this	
% /Title ()
% Put your actual complete title (no codes, scripts, shortcuts, or LaTeX commands) within the parentheses in mixed case
% Leave the space between \Title and the beginning parenthesis alone
% /Author ()
% Put your actual complete list of authors (no codes, scripts, shortcuts, or LaTeX commands) within the parentheses in mixed case. 
% Each author should be only by a comma. If the name contains accents, remove them. If there are any LaTeX commands, 
% remove them. 

% DISALLOWED PACKAGES
% \usepackage{authblk} -- This package is specifically forbidden
% \usepackage{balance} -- This package is specifically forbidden
% \usepackage{caption} -- This package is specifically forbidden
% \usepackage{color (if used in text)
% \usepackage{CJK} -- This package is specifically forbidden
% \usepackage{float} -- This package is specifically forbidden
% \usepackage{flushend} -- This package is specifically forbidden
% \usepackage{fontenc} -- This package is specifically forbidden
% \usepackage{fullpage} -- This package is specifically forbidden
% \usepackage{geometry} -- This package is specifically forbidden
% \usepackage{grffile} -- This package is specifically forbidden
% \usepackage{hyperref} -- This package is specifically forbidden
% \usepackage{navigator} -- This package is specifically forbidden
% (or any other package that embeds links such as navigator or hyperref)
% \indentfirst} -- This package is specifically forbidden
% \layout} -- This package is specifically forbidden
% \multicol} -- This package is specifically forbidden
% \nameref} -- This package is specifically forbidden
% \natbib} -- This package is specifically forbidden -- use the following workaround:
% \usepackage{savetrees} -- This package is specifically forbidden
% \usepackage{setspace} -- This package is specifically forbidden
% \usepackage{stfloats} -- This package is specifically forbidden
% \usepackage{tabu} -- This package is specifically forbidden
% \usepackage{titlesec} -- This package is specifically forbidden
% \usepackage{tocbibind} -- This package is specifically forbidden
% \usepackage{ulem} -- This package is specifically forbidden
% \usepackage{wrapfig} -- This package is specifically forbidden
% DISALLOWED COMMANDS
% \nocopyright -- Your paper will not be published if you use this command
% \addtolength -- This command may not be used
% \balance -- This command may not be used
% \baselinestretch -- Your paper will not be published if you use this command
% \clearpage -- No page breaks of any kind may be used for the final version of your paper
% \columnsep -- This command may not be used
% \newpage -- No page breaks of any kind may be used for the final version of your paper
% \pagebreak -- No page breaks of any kind may be used for the final version of your paperr
% \pagestyle -- This command may not be used
% \tiny -- This is not an acceptable font size.
% \vspace{- -- No negative value may be used in proximity of a caption, figure, table, section, subsection, subsubsection, or reference
% \vskip{- -- No negative value may be used to alter spacing above or below a caption, figure, table, section, subsection, subsubsection, or reference

\setcounter{secnumdepth}{0} %May be changed to 1 or 2 if section numbers are desired.

%\bibliographystyle{aaai}
\bibliography{citation}

% The file aaai20.sty is the style file for AAAI Press 
% proceedings, working notes, and technical reports.
%
\setlength\titlebox{2.5in} % If your paper contains an overfull \vbox too high warning at the beginning of the document, use this
% command to correct it. You may not alter the value below 2.5 in
\title{Prediction through Genetic Metaheuristic algorithm \\ and Long Short Term Memory based on Covid-19}
%Your title must be in mixed case, not sentence case. 
% That means all verbs (including short verbs like be, is, using,and go), 
% nouns, adverbs, adjectives should be capitalized, including both words in hyphenated terms, while
% articles, conjunctions, and prepositions are lower case unless they
% directly follow a colon or long dash
\author{ University of Padua \\ Department of Information Engineering \\ Intellingent Systems Course \\ Date 2019-20 \\ % All authors must be in the same font size and format. Use \Large and \textbf to achieve this result when breaking a line
%If you have multiple authors and multiple affiliations
% use superscripts in text and roman font to identify them. For example, Sunil Issar,\textsuperscript{\rm 2} J. Scott Penberthy\textsuperscript{\rm 3} George Ferguson,\textsuperscript{\rm 4} Hans Guesgen\textsuperscript{\rm 5}. Note that the comma should be placed BEFORE the superscript for optimum readability
Luca Perali, Michele Thiella
}
 \begin{document}

\maketitle

\begin{abstract}
	In recent years, Machine Learning (ML) has been gaining unprecedented visibility and success in the history of artificial intelligence, thanks certainly to the wide range of problems in which it can be applied: Machine Vision, Keyword Spotting Analysis, Data Sequence Prediction, etc. \\
	In the training phase of the ML algorithms, the programmer have to find good hyperparameters capable of obtaining the best results in the test phase. Due to the exponential number of possible combination of hyperparameters, this process can be time consuming. As these hyperparameters strongly depends on the type of problem being faced and on the dataset used, it is not possible to share parameters values between problems or models. \\
	In this project we wanted to deepen a metaheuristic method based on evolutionary genetics, named Coronavirus Optimization Algorithm (CVOA) \cite{martnezlvarez2020coronavirus}, which is a metaheuristics inspired by the virus spread within the population, using it for the automatic calculation of hyperparameters for the training phase of an ML algorithm. This approach would like to reduce the time needed to find good hyperparameters to train the model for a specific problem.
	
\end{abstract}

\section{Introduction}
	While developing a ML algorithm, as usual, it's required to define the structure of a model and the parameter to train that model in a dataset. Once defined the parameter to tune, and their domain, there is usually a huge number of configuration of the parameters to simply try all of them and see which is the best, therefore different metaheuristics try to explore the hyperparameter search space in a clever way. \\
	In the first part of this report there is a short description of the LSTM model and the genetic algorithm to refresh the basic concept. Then there is a description of the CVOA metaheuristic algorithm which will be combined with the LSTM for testing purposes, knowing that every model with appropriate encoding could potentially replace the LSTM. In the end there are some consideration about the efficacy of the method and the conclusion. 
	
	
	This algorithm, as described in the referenced paper, promise very strong advantages:
	\begin{itemize}
		\item the Covid-19 statistics of propagation are well known from the scientific community therefore there is no need to tune them;
		\item the spread automatically stops after several iteration, with no need to be configured; 
		\item the coronavirus high spreading rate is useful for exploring promising regions more
		thoroughly (intensification) while the use of parallel strains ensures that all regions
		of the search space are evenly explored (diversification).
	\end{itemize}

\section{Basic Concepts}
CVOA is an algorithm used to tune the parameter of a neural network (NN), therefore let's describe the NN model used for the project and the data this net want to fit.
It has been decided to train a LSTM model with a dataset of sequences representing the number of infected of a virus (in our case again the Covid19) over the time. Here is needed to specify that either the metaheuristic or the LSTM deal with Covid19, however it is involved in two independent way:
\begin{itemize}
	\item the \textbf{CVOA is inspired from the Covid19}, simulating the generation of new individual with the spread of new infected;
	\item the \textbf{LSTM model is trained on the Covid19 data} of infected over the time, therefore can be used to forecast the number of infected.
\end{itemize}


\subsection{Long Short Term Memory}
The LSTM model is a neural network based on a computational unit (Figure \ref{fig:computational_unit}) that can memorize the context (represented by the function $ c(t) $). Each unit receive as input the concatenation of $ x(t)h(t-1)c(t-1) $ where $ h(t) = x(t-1) $ and $ x(t) $ is the input from dataset.
\begin{figure}[!h]
	\includegraphics[width=\columnwidth]{img/computational_unit}
	\caption{The computational unit of a LSTM with input from dataset $ x(t) $, output $ h(t) $ and context function $ c(t) $. The input of the unit at time $ t $ is $ x(t)h(t-1)c(t-1) $ .}
	\label{fig:computational_unit}
\end{figure}


\begin{figure}[!h]
	\includegraphics[width=\columnwidth]{img/LSTM}
	\caption{The general structure of the LSTM used in the experiment}
	\label{fig:LSTM}
\end{figure}
A LSTM neural network is made by at least one layer of LSTM units. The dataset to train a LSTM network in general is made by a sequence of input data and a corresponding sequence of output data.
In our specific case the dataset is time sequence of the number of infected of Covid19 in different country.\\ The structure of NN is as in Figure \ref{fig:LSTM}.\\ 
In Table \ref{tab:params} there are the tuned hyperparameters values chosen for the experiment. Note that the total number of possible configuration of hyperparameters is $ 12*6*8*6 = 3456 $ however as will be shown, the number of trained model will be much less, decreasing the time to find good hyperparameters.\\
\begin{table}[!h]
	\caption{Table of the values to tune for the model. The chosen value for each parameter are: number of units per layer $ L_i $, Learning Rate ($ LR $), dropout percentage ($ D $) and number of LSTM layers ($ N $). }
	\begin{tabular}{ccccc}
		   &$L_i$ &   $ LR $ 	& $D $ & $N$\\ 
	\hline
		1  & 25   & $ 10^{-1} $	& 0.05 & 1	\\
		2  & 50   & $ 10^{-2} $ & 0.10 & 2 	\\
		3  & 75   & $ 10^{-3} $ & 0.15 & 3 	\\
		4  & 100  & $ 10^{-4} $ & 0.20 & 4 	\\
		5  & 125  & $ 10^{-5} $ & 0.25 & 5 	\\
		6  & 150  & $ 10^{-6} $ & 0.30 & 6 	\\
		7  & 175  &  			& 0.35 &   	\\
		8  & 200  &  			& 0.40 &   	\\
		9  & 225  &  			&      &   	\\
		10 & 250  &  			&      &   	\\
		11 & 275  &  			&      &   	\\
		12 & 300  &  			&      &   	\\
	\end{tabular}

	\label{tab:params}
\end{table}

Let's now analyze the encoding method to obtain the individual of the genetic algorithm from the LSTM trained model. Each trained model can be identified with its training hyperparameters value, moreover there are some of them which are set in every model (fixed parameters) and other that are not (variable parameter). In Figure \ref{fig:individual} it's shown the general definition of individual of CVOA genetic algorithm. As an example, if a model is trained with $ LR = 10^{-1}, D = 0.4, N = 2, L_1 = 50, L_2 = 25 $ the corresponding individual would be represented by $ [1, 8, 2] + [2, 1] $. Note that the value inside the parenthesis are the index of the corresponding value in Table \ref{tab:params}. 

\begin{figure}
	\caption{The representation of a LSTM as individual of the GA. Learning Rate ($ LR $), Dropout ($ D $ and Number of LSTM layer ($ N $) compose the fixed part of the individual and the number of LSTM unit ($ L_i $) for each LSTM layer ($ i $) represent the variable part. Note that the "+" symbol stand for concatenation.}
	\includegraphics[width=\columnwidth]{img/individual}
	\label{fig:individual}
\end{figure}


\subsection{Genetic Algorithm}
	The genetic part of CVOA is built by following a standard Genetic Algorithm (GA) but adding and modifying some components.
The main components and functions performed by the GA are folded successively:

\subsubsection{GA Constant Parameters}
This subsection presents the main parameters used and which will be mentioned in the following sections.
All probabilities involved in the GA are based on the real data of Coronavirus propagation model.
\begin{itemize}
\item $_PDIE$: An infected individual can die with a given probability.
\item $_PSUPERSPREADER$: It is the probability that an individual spread the disease to a greater number of healthy individuals. After this condition is validated, two situations can be found: % TODO: cercare di fixare l'underscore _ che è osceno così e magari cambia il font dei nomi (senti fish)
\subitem $ORDINARY\_RATE$: It is a standard infected individual (not a super-spreader) and this rate correspondes to its infection rate
\subitem $SUPERSPREADER\_RATE$: If the infected individual turns out to be super-spreader, this is its infection rate which is at least more than double of Ordinary-Rate
\item $_PREINFECTION$: Probability that a recovered individual can be re-infected.
\item $_PISOLATION$: It corresponds to the probability of applying measures for social isolation to an individual. This parameter helps to reduce the exponential growth of the infected population after each iteration. Therefore, a high value must be assigned to this probability. Note that this value is not well defined by the real pandemic data because different countries have made different choices about it. 
\item $_PTRAVEL$: This probability simulates how an infected individual can travel to any place in the world and can infect healthy individuals.
\item $PANDEMIC\_DURATION$: This parameter simulates the duration of the pandemic.
\end{itemize}

\subsubsection{Population Individuals}
	As explained above, figure \ref{fig:individual}, the individuals of the population are composed of two arrays of integer values (fixed and variable length). They are also divided into three categories:
\begin{itemize}
\item Dead
\item Recovered
\item Infected
\end{itemize}
There is another category in which only some infected individuals can belong, according to a certain probability ($_PSUPERSPREADER$). Only individuals who have the best fitness value (MAPE) of the population have a chance to fall into this category. They are called Super-Spreaders and have a higher infection potential than other individuals. In this way the evolution of infected individuals of the following epoch will be based on individuals with good MAPE value and will be statistically better.

\subsubsection{Population}
	The population divides individuals into three different (sub)populations, managed through lists and updated and maintained during the whole algorithm, in according with the type of individuals:
\begin{itemize}
\item Dead: It contains dead individuals. They can no longer be used in the algorithm.
\item Recovered: After each epoch, infected individuals are removed from their population and inserted here. These individuals can return to influence the algorithm if they are reinfected ($_PREINFECTION$). Also individuals who simulate social distancing measures and therefore are isolated, belong to this population ($_PISOLATION$).
\item New-Infected: It holds all the new individuals that have been infected in the previous epoch.
\end{itemize}

\subsubsection{Main Steps} \label{mainSteps}
\begin{itemize}
\item Patient-Zero: The initial population consists of only one individual, called patient-zero (PZ). As in the real coronavirus epidemic, it identifies the first human being infected.
\item Fitness function: The fitness function, calculated for each infected individual at each epoch corresponds to the execution of the LSTM training, which structure is specified by the hyper-parameters given by the individuals themselves.
\item Fitness Value: The fitness value of each individual is the Mean Absolute Percentage Error (MAPE) value. It is returned as a result of performing the LSTM training and tries to quantify the goodness of the hyper-parameters used, as it corresponds to the measure of the accuracy of the forecast of a forecasting method with respect to the real value.
\item Mutation: The basic process for the mutation is the single position mutation. It corresponds to the change of a value of a specific element in the fixed or variable array of an individual.
First, a signed change amount \begin{math}C \in \{-2 ,-1, +1, +2\}\end{math} is randomly determined using the following criteria. A random real number $P$ between $0$ and $1$ is generated using a uniform distribution. The change amount will be:
\begin{itemize}
\item \begin{math}C=-2 \iff P<0,25\end{math}
\item \begin{math}C=-1 \iff P<0,50\end{math}
\item \begin{math}C=+1 \iff P<0,75\end{math}
\item \begin{math}C=+2 \iff P \leq 1\end{math}
\end{itemize}
Once the amount of change is determined, the new value for the infected element is computed. If its previous value is $V$ , then the new value after the single position mutation will be \begin{math}V\textsuperscript{1} = V + C\end{math}. If the new value $V\textsuperscript{1}$ exceeds the limits defined for the individual codification, such value is set to the maximum or minimum allowed value accordingly.
%• Numerous mutations
\item Crossover: The evolution of the population, as in the real case, is not given by the reproduction of individuals, but the virus spreads from one individual to another. In this sense there is no crossover step, in its place a Propagation Function has been developed.
\item Propagation Function:
Some of the infected individuals die ($_PDIE$). Such individuals can no longer infect new individuals.
The individuals surviving the coronavirus in an epoch will infect new individuals (intensification). According to the probability $_PSUPERSPREADER$, an individual can spread the desease with two rate:
\subitem Ordinary spreader, with spreading rate given by $ORDINARY\_RATE$.
\subitem Super-spreaders, with spreading rate given by $SUPERSPREADER\_RATE$.\\
To ensure diversification, both types of spreader individuals can travel and explore solutions quite dissimilar ($_PTRAVEL$), thus allowing to propagate the disease to solutions that may be quite different. In case of not being traveler, new solutions will change according to an $ORDINARY\_RATE$.
\item Stop Criterion:
The stop criterion of this algorithm, by construction, does not need to be controlled by any parameter. The recovered and dead populations are constantly growing as time goes by, and the new infected population cannot infect new individuals. It is expected that the number of infected individuals increases for a certain number of iterations. However, from a particular iteration on, the size of the new infected population will be smaller than that of the current one because recovered and dead populations are too big, and the size of the infected population decays over time. \\
Additionally, due to time constraint, it is possible to set up a maximum number of iterations ($PANDEMIC\_DURATION$) to force the stop criterion. The social isolation measures also contributes to achieve it.
\end{itemize}

\section{CVOA Meta-heuristic}
Including the basic concepts, % TODO: aggiungere riferimenti alle sezioni che vengono richiamate
 it is now possible to understand the overall functioning of the CVOA and analyze it in a more specific way.

\subsection{CVOA Main Method}
The main method of the algorithm is as follows:
\lstset{language=Python}
\lstset{frame=lines}
\lstset{caption={CVOA main method}}
\lstset{label={lst:code_direct}}
\lstset{basicstyle=\footnotesize}
\lstset{showstringspaces=false}
\begin{lstlisting}[mathescape=true]
PZ = {fixed part} + {var part}
while (PANDEMIC_DURATION $\geq$ 0 && epidemic) {
    propagateDisease()
    getBestIndividual()
    if (infected_population is empty)
        epidemic = false
    PANDEMIC_DURATION --
}
\end{lstlisting}
As mentioned in section \ref{mainSteps}, the initial population is composed of a single individual (PZ), with values of the arrays that compose it randomly generated (within their limits).Then a loop is performed until the stop criterion is reached. Each iteration of the cycle corresponds to an epoch of the GA, in each epoch, initially the disease is propagated and then the best current individual is found. Once the stop conditions are updated, the cycle starts again.

\section{Experiment Evaluation}

\section{Discussion of Differences}

\section{Conclusion}

\subsection{Future Work}

\clearpage
\onecolumn
\section{Pseudo-Python Code}
\lstset{language=Python}
\lstset{frame=lines}
\lstset{caption={propagateDisease method}}
\lstset{label={lst:code_direct}}
\lstset{basicstyle=\footnotesize}
\lstset{showstringspaces=false}
\lstset{numbers=left}					% show row number (to use them in code?)
\lstset{stepnumber=1}
\begin{lstlisting}[mathescape=true]
Define PZ
for (individual in infected_population):
    individual.fitness = fitness_function(individual)

infected = sorted(infected, key=lambda i: i.fitness)
best_solution = population[0]
idx_super_spreader = S$\textsubscript{\%}$ * len(infected)
idx_deaths = D$\textsubscript{\%}$ * len(infected)

for (individual in infected_population):
    if (idx_individual $\geq$ idx_deaths):
        infected_population.remove(individual)
        dead_population.add(individual)
    else:
        if (idx_individual < idx_super_spreader):
            nInfected = MIN_SUPERSPREAD + random.randint(0, MAX_SUPERSPREAD - MIN_SUPERSPREAD)
        else:
            nInfected = random.randint(0, MAX_SPREAD)
            if (individual is traveler):
                traverl_distance = -1
            else: 
                traverl_distance = 1
        while (nInfected > 0):
            new_infected = individual.infect(travel_distance)
            if (new_infected $\notin$ deaths || infected || new_infected_list || recovered): 
                new_infected_list.add(new_infected)
            if (new_infected $\in$ recovered && new_infected $\notin$ new_infected_list): 
                if (random.random() < $_PREINFECTION$): 
                    new_infected_list.add(new_infected)
                    recovered.remove(new_infected)
            nInfected --
        recovered.add(infected) 
        infected = new_infected_list
\end{lstlisting}
\vspace{5mm} %5mm vertical space
\lstset{language=Python}
\lstset{frame=lines}
\lstset{caption={individual.infected method}}
\lstset{label={lst:code_direct}}
\lstset{basicstyle=\footnotesize}
\lstset{showstringspaces=false}
\lstset{numbers=left}				
\lstset{stepnumber=1}
\begin{lstlisting}[mathescape=true]
mutated = deepcopy(individual)
if (random.randin(0, 2) > $\frac{1}{3}$)
    value = mutated.single_position_mutation(individual.fixed_pard[2])
    mutated.var_part.resize(value)
    travel_distance = -1 

total_size = len(mutated.fixed_part) + len(mutated.var_part) 
if (travel_distance < 0): 
    nMutated = random.randint (0, total_size-1)
else: 
    nMutated = 1

count = 0
mutated_positions = []
while (count < nMutated):
    pos = random.randint(0, total_size-1)
    if (pos $\notin$ mutated_positions && pos $\neq$ 2)
        if (pos < mutated.fixed_part - 1):
            max_value, min_value = MAX_VAL_FIXED_PART, MIN_VAL_FIXED_PART
        else:
            max_value, min_value = MAX_VAL_VAR_PART, MIN_VAL_VAR_PART
        mutated_position[pos] = random.randint(min, max)

    count ++
\end{lstlisting}
\clearpage
\twocolumn

\printbibliography

\end{document}
